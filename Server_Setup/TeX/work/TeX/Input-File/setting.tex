\usepackage[dvipdfmx]{graphicx}
\usepackage{amsmath,amssymb}    %数式関連のパッケージ
\usepackage{bm}                 %数式(ベクトル)関連のパッケージ
\usepackage{ascmac}             %枠囲のためのパッケージ
\usepackage{txfonts}            %Times,Helvetica用のパッケージ {times}より推奨
\usepackage{float}              %図の貼り付けの{H}(強制的に、ここ)が指定できる。{here}と同等

%pdf用の目次等のセッティング
\usepackage[dvipdfmx]{hyperref}
\usepackage{pxjahyper}
\hypersetup{%
   bookmarksnumbered=true,%
   bookmarkstype=toc,%
   pdftitle={タイトル},%
   pdfsubject={TeXテンプレート},%
   pdfauthor={氏名},%
   pdfkeywords={TeX,PDF,テンプレート}}

%フォントの変更
\usepackage[uplatex]{otf}
\renewcommand{\rmdefault}{ptm} %デフォルトのローマンフォントをTimesに変える。
\renewcommand{\sfdefault}{phv} %デフォルトのサンセリフフォントをHelveticaに変える。
%見出しのフォントの設定
\renewcommand{\headfont}{\bfseries\mcfamily\rmfamily}

%新コマンド
\newcommand{\divergence}{\mathrm{div}\,}  %ダイバージェンス
\newcommand{\grad}{\mathrm{grad}\,}  %グラディエント
\newcommand{\rot}{\mathrm{rot}\,}  %ローテーション
\def\vector#1{\mbox{\boldmath $#1$}} %ベクトル

\makeatletter
% 数式番号に章番号を追加
\@addtoreset{equation}{section}
\def\theequation{\thesection.\arabic{equation}}

\def\@thesis{ }
\def\id#1{\def\@id{#1}}
\def\department#1{\def\@department{#1}}

\makeatother

\makeatletter
\def\@maketitle{
\begin{center}
\vspace{20mm}
{\LARGE \@thesis \par} %表題
\vspace{15mm}
{\Huge \@title \par}% タイトル部分
\vspace{120mm}
{\Large \@department \par}	% 所属部分
\vspace{5mm}
{\Large \@author\par}% 氏名
\vspace{10mm}
{\Large \@date\par}	 % 年月日部分
\end{center}
\par\vskip 1.5em
}
\makeatother

%\title{タイトル}
%\department{所属1}
%\author{氏名}
%\date{\today}
